\documentclass{article}
\usepackage{hyperref}
\usepackage{enumitem}

\title{Final Project Proposal}
\author{Raymond Blaha and Paul Hwang}
\date{\today}

\begin{document}

\maketitle

\section{Introduction / Question}
Can we predict wildfire risk zones using historical wildfire, vegetation, and climate data in the califorina region?

\section{Data}
The data for this study will be collated from the following sources (At the current moment. Subject to change.):
\begin{enumerate}
  \item Historical Wildfire Data: \url{https://modis.gsfc.nasa.gov/data/dataprod/mod14.php}
  \item Vegetation Data: \url{https://www.usgs.gov/core-science-systems/science-analytics-and-synthesis/gap/science/land-cover-data-download?qt-science_center_objects=0#qt-science_center_objects}
  \item Climate Data: \url{https://www.ncdc.noaa.gov/cdo-web/}
\end{enumerate}

\section{Project Outline}

\subsection{Objective}
identifying high-risk wildfire zones using historical data. 

To approach this problem, an autoencoder will be used to extract key features from complex data, 
and a CNN will analyze spatial correlations from the satellite imagery. 
Finally, we can use the LSTM to take into account the temporal climate instances to predict wildfire risk zones.

\subsection{Data Visualization}
Data will be visualized to directly support the modeling process, using Python libraries to:
\begin{enumerate}
    \item Display extracted features from the autoencoder.
    \item Visualize patterns identified by the CNN and present the predictions.
    \item Visualize the LSTM predictions given the historical climate data.
\end{enumerate}

\subsection{Modeling}
\subsubsection{Autoencoder}
The autoencoder will be used to extract key features from complex data,  such as satellite imagery, to reduce noise and improve model performance.

\subsubsection{CNN}
The CNN will then analyze these features to identify spatial patterns that will be key for predicting wildfire risk zones.

\subsection{LSTM}
The LSTM will be used to take into account the temporal climate instances which will give the model more context to predict wildfire risk zones.

\subsection{Visualization of Results}
Results will be presented in a simple format, such as risk zone maps based on the CNN analysis.

\section{Conclusion}
This project will yield:
\begin{enumerate}
    \item An autoencoder-CNN-LSTM integrated model for wildfire risk prediction.
    \item Clear visualizations that illustrate the model's predictive capabilities.
\end{enumerate}

\end{document}